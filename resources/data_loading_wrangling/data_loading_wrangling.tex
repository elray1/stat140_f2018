\documentclass[]{article}
\usepackage{lmodern}
\usepackage{amssymb,amsmath}
\usepackage{ifxetex,ifluatex}
\usepackage{fixltx2e} % provides \textsubscript
\ifnum 0\ifxetex 1\fi\ifluatex 1\fi=0 % if pdftex
  \usepackage[T1]{fontenc}
  \usepackage[utf8]{inputenc}
\else % if luatex or xelatex
  \ifxetex
    \usepackage{mathspec}
  \else
    \usepackage{fontspec}
  \fi
  \defaultfontfeatures{Ligatures=TeX,Scale=MatchLowercase}
\fi
% use upquote if available, for straight quotes in verbatim environments
\IfFileExists{upquote.sty}{\usepackage{upquote}}{}
% use microtype if available
\IfFileExists{microtype.sty}{%
\usepackage{microtype}
\UseMicrotypeSet[protrusion]{basicmath} % disable protrusion for tt fonts
}{}
\usepackage[margin=1.5cm]{geometry}
\usepackage{hyperref}
\hypersetup{unicode=true,
            pdftitle={Stat 140: R Commands for Loading and Wrangling Data},
            pdfborder={0 0 0},
            breaklinks=true}
\urlstyle{same}  % don't use monospace font for urls
\usepackage{color}
\usepackage{fancyvrb}
\newcommand{\VerbBar}{|}
\newcommand{\VERB}{\Verb[commandchars=\\\{\}]}
\DefineVerbatimEnvironment{Highlighting}{Verbatim}{commandchars=\\\{\}}
% Add ',fontsize=\small' for more characters per line
\usepackage{framed}
\definecolor{shadecolor}{RGB}{248,248,248}
\newenvironment{Shaded}{\begin{snugshade}}{\end{snugshade}}
\newcommand{\KeywordTok}[1]{\textcolor[rgb]{0.13,0.29,0.53}{\textbf{#1}}}
\newcommand{\DataTypeTok}[1]{\textcolor[rgb]{0.13,0.29,0.53}{#1}}
\newcommand{\DecValTok}[1]{\textcolor[rgb]{0.00,0.00,0.81}{#1}}
\newcommand{\BaseNTok}[1]{\textcolor[rgb]{0.00,0.00,0.81}{#1}}
\newcommand{\FloatTok}[1]{\textcolor[rgb]{0.00,0.00,0.81}{#1}}
\newcommand{\ConstantTok}[1]{\textcolor[rgb]{0.00,0.00,0.00}{#1}}
\newcommand{\CharTok}[1]{\textcolor[rgb]{0.31,0.60,0.02}{#1}}
\newcommand{\SpecialCharTok}[1]{\textcolor[rgb]{0.00,0.00,0.00}{#1}}
\newcommand{\StringTok}[1]{\textcolor[rgb]{0.31,0.60,0.02}{#1}}
\newcommand{\VerbatimStringTok}[1]{\textcolor[rgb]{0.31,0.60,0.02}{#1}}
\newcommand{\SpecialStringTok}[1]{\textcolor[rgb]{0.31,0.60,0.02}{#1}}
\newcommand{\ImportTok}[1]{#1}
\newcommand{\CommentTok}[1]{\textcolor[rgb]{0.56,0.35,0.01}{\textit{#1}}}
\newcommand{\DocumentationTok}[1]{\textcolor[rgb]{0.56,0.35,0.01}{\textbf{\textit{#1}}}}
\newcommand{\AnnotationTok}[1]{\textcolor[rgb]{0.56,0.35,0.01}{\textbf{\textit{#1}}}}
\newcommand{\CommentVarTok}[1]{\textcolor[rgb]{0.56,0.35,0.01}{\textbf{\textit{#1}}}}
\newcommand{\OtherTok}[1]{\textcolor[rgb]{0.56,0.35,0.01}{#1}}
\newcommand{\FunctionTok}[1]{\textcolor[rgb]{0.00,0.00,0.00}{#1}}
\newcommand{\VariableTok}[1]{\textcolor[rgb]{0.00,0.00,0.00}{#1}}
\newcommand{\ControlFlowTok}[1]{\textcolor[rgb]{0.13,0.29,0.53}{\textbf{#1}}}
\newcommand{\OperatorTok}[1]{\textcolor[rgb]{0.81,0.36,0.00}{\textbf{#1}}}
\newcommand{\BuiltInTok}[1]{#1}
\newcommand{\ExtensionTok}[1]{#1}
\newcommand{\PreprocessorTok}[1]{\textcolor[rgb]{0.56,0.35,0.01}{\textit{#1}}}
\newcommand{\AttributeTok}[1]{\textcolor[rgb]{0.77,0.63,0.00}{#1}}
\newcommand{\RegionMarkerTok}[1]{#1}
\newcommand{\InformationTok}[1]{\textcolor[rgb]{0.56,0.35,0.01}{\textbf{\textit{#1}}}}
\newcommand{\WarningTok}[1]{\textcolor[rgb]{0.56,0.35,0.01}{\textbf{\textit{#1}}}}
\newcommand{\AlertTok}[1]{\textcolor[rgb]{0.94,0.16,0.16}{#1}}
\newcommand{\ErrorTok}[1]{\textcolor[rgb]{0.64,0.00,0.00}{\textbf{#1}}}
\newcommand{\NormalTok}[1]{#1}
\usepackage{graphicx,grffile}
\makeatletter
\def\maxwidth{\ifdim\Gin@nat@width>\linewidth\linewidth\else\Gin@nat@width\fi}
\def\maxheight{\ifdim\Gin@nat@height>\textheight\textheight\else\Gin@nat@height\fi}
\makeatother
% Scale images if necessary, so that they will not overflow the page
% margins by default, and it is still possible to overwrite the defaults
% using explicit options in \includegraphics[width, height, ...]{}
\setkeys{Gin}{width=\maxwidth,height=\maxheight,keepaspectratio}
\IfFileExists{parskip.sty}{%
\usepackage{parskip}
}{% else
\setlength{\parindent}{0pt}
\setlength{\parskip}{6pt plus 2pt minus 1pt}
}
\setlength{\emergencystretch}{3em}  % prevent overfull lines
\providecommand{\tightlist}{%
  \setlength{\itemsep}{0pt}\setlength{\parskip}{0pt}}
\setcounter{secnumdepth}{0}
% Redefines (sub)paragraphs to behave more like sections
\ifx\paragraph\undefined\else
\let\oldparagraph\paragraph
\renewcommand{\paragraph}[1]{\oldparagraph{#1}\mbox{}}
\fi
\ifx\subparagraph\undefined\else
\let\oldsubparagraph\subparagraph
\renewcommand{\subparagraph}[1]{\oldsubparagraph{#1}\mbox{}}
\fi

%%% Use protect on footnotes to avoid problems with footnotes in titles
\let\rmarkdownfootnote\footnote%
\def\footnote{\protect\rmarkdownfootnote}

%%% Change title format to be more compact
\usepackage{titling}

% Create subtitle command for use in maketitle
\newcommand{\subtitle}[1]{
  \posttitle{
    \begin{center}\large#1\end{center}
    }
}

\setlength{\droptitle}{-2em}

  \title{Stat 140: R Commands for Loading and Wrangling Data}
    \pretitle{\vspace{\droptitle}\centering\huge}
  \posttitle{\par}
    \author{}
    \preauthor{}\postauthor{}
    \date{}
    \predate{}\postdate{}
  
\usepackage{booktabs}

\begin{document}
\maketitle

\begin{table}[!h]
\begin{tabular}{p{4.25cm} p{12.5cm}}
\toprule
\multicolumn{2}{l}{\textbf{Loading Data and Taking a First Look}} \\
\midrule
Load a package & \verb&library(dplyr)& \\
\midrule
Assign a value to a variable & \verb&my_var <- 3& \\
\midrule
Read a csv file & \verb&nhanes <- read_csv("path/to/nhanes.csv")& \\
\midrule
First few lines of data frame & \verb&head(nhanes)& \\
\midrule
Structure of data frame & \verb&str(nhanes)& \\
\midrule
Dimensions of data frame & \verb&dim(nhanes)&, \verb&nrow(nhanes)&, \verb&ncol(nhanes)& \\
\midrule
\multicolumn{2}{l}{\textbf{Working with Categorical Variables}} \\
\midrule
Convert a nominal categorical variable to factor & \verb&nhanes <- nhanes %>% mutate(Gender = factor(Gender))& \\
\midrule
Convert an ordinal & \verb&nhanes <- nhanes %>% mutate(& \\
categorical variable & \verb&  Education = factor(Education,& \\
to factor & \verb&    levels = c("High School", "Some College", "College Grad"),& \\
 & \verb&    ordered = TRUE))& \\
\midrule
View the distinct values of a variable (mainly useful for categorical variables) & \verb&nhanes %>% distinct(Education)& \\
\midrule
Count number of obs. in each level of a categorical variable & \verb&nhanes %>% count(Education)& \\
\midrule
Count number of obs. & \verb&nhanes %>% count(Education, Gender) %>%& \\
in each combination of levels of two categorical var's & \verb&  spread(Gender, n)& \\
\midrule
\multicolumn{2}{l}{\textbf{Summarizing Quantitative Variables}} \\
\midrule
Mean of quantitative      & \verb&nhanes %>%& \\
variable, for each level  & \verb&  group_by(MaritalStatus) %>%& \\
of a categorical variable & \verb&  summarize(mean_poverty_index = mean(Poverty, na.rm = TRUE))& \\
\midrule
\multicolumn{2}{l}{\textbf{Data Wrangling}} \\
\midrule
Add or modify a variable & \verb&nhanes_modified <- nhanes %>%& \\
in a data frame & \verb&  mutate(Weight_pounds = Weight * 2.205)& \\
\midrule
Filter observational units, & \verb&nhanes_fewer_obs_units <- nhanes %>%& \\
character condition & \verb&  filter(Education == "High School")& \\
\midrule
Filter, character condition & \verb&nhanes_fewer_obs_units <- nhanes %>%& \\
with multiple values & \verb&  filter(Education == "High School" | Education == "Some College") & \\
\midrule
Filter, numeric condition & \verb&nhanes_fewer_obs_units <- nhanes %>%& \\
 & \verb&  filter(Age >= 22)& \\
\midrule
Filter, multiple conditions & \verb&nhanes_fewer_obs_units <- nhanes %>%& \\
 & \verb&  filter(Education == "High School", Age >= 22)& \\
\midrule
Sort, ascending order & \verb&nhanes_sorted <- nhanes %>% arrange(Age)& \\
\midrule
Sort, descending order & \verb&nhanes_sorted <- nhanes %>% arrange(desc(Age))& \\
\bottomrule
\end{tabular}
\end{table}

\newpage

In this document I am going to summarize the main commands and concepts
for R that we have learned so far -- along with a couple of others that
you haven't seen but are closely related to what we've done so far.
These are organized into four main groups:

\begin{enumerate}
\def\labelenumi{\arabic{enumi}.}
\tightlist
\item
  R variables and the assignment operator
\item
  Basic interactions with data frames

  \begin{enumerate}
  \def\labelenumii{\alph{enumii}.}
  \tightlist
  \item
    Reading data into R from spreadsheet files
  \item
    Getting a first look at what's in a data frame
  \item
    Converting categorical variables to factors
  \end{enumerate}
\item
  Summarizing categorical variables
\item
  Summarizing quantitative variables
\item
  Data wrangling
\end{enumerate}

I will illustrate the ideas using the NHANES data we looked at in Lab 1.

\subsection{1. R variables and the assignment
operator}\label{r-variables-and-the-assignment-operator}

In R, we use the word ``variable'' in two ways. The first is a name that
we've given a value that we want to be able to re-use later. In the
example below, \texttt{my\_var} is a variable. We have \emph{assigned}
the value 3 to it using the \emph{assignment operator},
\texttt{\textless{}-} (a less than sign followed by a minus sign, to
form an arrow).

\begin{Shaded}
\begin{Highlighting}[]
\NormalTok{my_var <-}\StringTok{ }\DecValTok{3}
\end{Highlighting}
\end{Shaded}

We can see the value that's currently assigned to \texttt{my\_var} by
entering the name of the variable on its own line:

\begin{Shaded}
\begin{Highlighting}[]
\NormalTok{my_var}
\end{Highlighting}
\end{Shaded}

\begin{verbatim}
## [1] 3
\end{verbatim}

We can also use that value in later calculations:

\begin{Shaded}
\begin{Highlighting}[]
\NormalTok{my_var }\OperatorTok{*}\StringTok{ }\DecValTok{2}
\end{Highlighting}
\end{Shaded}

\begin{verbatim}
## [1] 6
\end{verbatim}

The second meaning of the word ``variable'' is more closely related to
our use of the word in statistics: a column in a data frame. We'll look
at that next.

\subsection{2. Basic interactions with data
frames}\label{basic-interactions-with-data-frames}

In R, the most common way to store data is in a data frame. You can
think of a data frame as being like a spreadsheet. Each row corresponds
to an observational unit, and each column corresponds to a variable.

\subsubsection{a. Reading data into R from spreadsheet
files}\label{a.-reading-data-into-r-from-spreadsheet-files}

Usually, the data are stored in a spreadsheet-like file outside of R.
The file format we'll work with most in this class is a csv file (csv
stands for comma separated value). We can read in csv files using the
\texttt{read\_csv} function, which is in the \texttt{readr} package:

\begin{Shaded}
\begin{Highlighting}[]
\KeywordTok{library}\NormalTok{(readr)}
\NormalTok{nhanes <-}\StringTok{ }\KeywordTok{read_csv}\NormalTok{(}\StringTok{"http://www.evanlray.com/data/misc/nhanes/nhanes.csv"}\NormalTok{)}
\end{Highlighting}
\end{Shaded}

\begin{verbatim}
## Parsed with column specification:
## cols(
##   ID = col_integer(),
##   Gender = col_character(),
##   Age = col_integer(),
##   Race = col_character(),
##   Education = col_character(),
##   MaritalStatus = col_character(),
##   HHIncome = col_character(),
##   Poverty = col_double(),
##   Weight = col_double(),
##   Length = col_double(),
##   Height = col_double(),
##   Diabetes = col_character(),
##   nPregnancies = col_integer(),
##   nBabies = col_integer(),
##   PregnantNow = col_character()
## )
\end{verbatim}

If the data file was stored on your computer instead of on the class
website, you would change the file location in these commands to where
the file is located on your computer.

There are also functions to read in data from other file formats. For
example, if your data were stored in an excel file (with a file
extension like xlsx), you could use the \texttt{read\_excel} function
from the \texttt{readxl} package to read the data in. This function
doesn't handle reading files from the internet very well yet, so we
won't use it much in this class -- but it's there if you need it later.

\subsubsection{b. Getting a first look at what's in the data
frame}\label{b.-getting-a-first-look-at-whats-in-the-data-frame}

There are a couple of questions I always ask myself whenever I'm
thinking about a new data set:

\begin{enumerate}
\def\labelenumi{\arabic{enumi}.}
\tightlist
\item
  How many observational units and variables are in this data set?
\item
  What are the variables and variable types?
\end{enumerate}

We've talked about three functions that can be used to help answer these
questions.

\paragraph{\texorpdfstring{\texttt{head}}{head}}\label{head}

The \texttt{head} function shows you the first few rows of the data set
(by default, the first 6 rows). It's good for getting a quick summary of
what's in the data frame, but it will not tell you how many
observational units there are.

\begin{Shaded}
\begin{Highlighting}[]
\KeywordTok{head}\NormalTok{(nhanes)}
\end{Highlighting}
\end{Shaded}

\begin{verbatim}
## # A tibble: 6 x 15
##      ID Gender   Age Race  Education MaritalStatus HHIncome Poverty Weight
##   <int> <chr>  <int> <chr> <chr>     <chr>         <chr>      <dbl>  <dbl>
## 1  3923 female    80 White High Sch~ Married       55000-6~    4.27   71.1
## 2  1548 male      42 Black 9 - 11th~ LivePartner   5000-99~    0.3   115. 
## 3  1205 male       4 Hisp~ <NA>      <NA>          25000-3~    0.78   19.7
## 4  1519 male      12 Black <NA>      <NA>          75000-9~    2.96   63.7
## 5  4148 male       1 Black <NA>      <NA>          15000-1~    0.67   11.7
## 6  1681 female    14 White <NA>      <NA>          25000-3~    1.52   71.6
## # ... with 6 more variables: Length <dbl>, Height <dbl>, Diabetes <chr>,
## #   nPregnancies <int>, nBabies <int>, PregnantNow <chr>
\end{verbatim}

\paragraph{\texorpdfstring{\texttt{str}}{str}}\label{str}

The \texttt{str} function will print out some more detailed information
about the data frame, including how many observational units and
variables there are, and the type of each variable -- but its output is
a little less well organized.

\begin{Shaded}
\begin{Highlighting}[]
\KeywordTok{str}\NormalTok{(nhanes)}
\end{Highlighting}
\end{Shaded}

\begin{verbatim}
## Classes 'tbl_df', 'tbl' and 'data.frame':    5000 obs. of  15 variables:
##  $ ID           : int  3923 1548 1205 1519 4148 1681 3710 3733 2552 373 ...
##  $ Gender       : chr  "female" "male" "male" "male" ...
##  $ Age          : int  80 42 4 12 1 14 56 0 33 46 ...
##  $ Race         : chr  "White" "Black" "Hispanic" "Black" ...
##  $ Education    : chr  "High School" "9 - 11th Grade" NA NA ...
##  $ MaritalStatus: chr  "Married" "LivePartner" NA NA ...
##  $ HHIncome     : chr  "55000-64999" "5000-9999" "25000-34999" "75000-99999" ...
##  $ Poverty      : num  4.27 0.3 0.78 2.96 0.67 1.52 5 2.46 NA 2.81 ...
##  $ Weight       : num  71.1 115.4 19.7 63.7 11.7 ...
##  $ Length       : num  NA NA NA NA 84.2 NA NA 61.7 NA NA ...
##  $ Height       : num  162 165 110 170 NA ...
##  $ Diabetes     : chr  "Yes" "Yes" "No" "No" ...
##  $ nPregnancies : int  5 NA NA NA NA NA 2 NA NA 3 ...
##  $ nBabies      : int  4 NA NA NA NA NA 2 NA NA 2 ...
##  $ PregnantNow  : chr  NA NA NA NA ...
##  - attr(*, "spec")=List of 2
##   ..$ cols   :List of 15
##   .. ..$ ID           : list()
##   .. .. ..- attr(*, "class")= chr  "collector_integer" "collector"
##   .. ..$ Gender       : list()
##   .. .. ..- attr(*, "class")= chr  "collector_character" "collector"
##   .. ..$ Age          : list()
##   .. .. ..- attr(*, "class")= chr  "collector_integer" "collector"
##   .. ..$ Race         : list()
##   .. .. ..- attr(*, "class")= chr  "collector_character" "collector"
##   .. ..$ Education    : list()
##   .. .. ..- attr(*, "class")= chr  "collector_character" "collector"
##   .. ..$ MaritalStatus: list()
##   .. .. ..- attr(*, "class")= chr  "collector_character" "collector"
##   .. ..$ HHIncome     : list()
##   .. .. ..- attr(*, "class")= chr  "collector_character" "collector"
##   .. ..$ Poverty      : list()
##   .. .. ..- attr(*, "class")= chr  "collector_double" "collector"
##   .. ..$ Weight       : list()
##   .. .. ..- attr(*, "class")= chr  "collector_double" "collector"
##   .. ..$ Length       : list()
##   .. .. ..- attr(*, "class")= chr  "collector_double" "collector"
##   .. ..$ Height       : list()
##   .. .. ..- attr(*, "class")= chr  "collector_double" "collector"
##   .. ..$ Diabetes     : list()
##   .. .. ..- attr(*, "class")= chr  "collector_character" "collector"
##   .. ..$ nPregnancies : list()
##   .. .. ..- attr(*, "class")= chr  "collector_integer" "collector"
##   .. ..$ nBabies      : list()
##   .. .. ..- attr(*, "class")= chr  "collector_integer" "collector"
##   .. ..$ PregnantNow  : list()
##   .. .. ..- attr(*, "class")= chr  "collector_character" "collector"
##   ..$ default: list()
##   .. ..- attr(*, "class")= chr  "collector_guess" "collector"
##   ..- attr(*, "class")= chr "col_spec"
\end{verbatim}

\paragraph{\texorpdfstring{\texttt{dim}, \texttt{nrow}, and
\texttt{ncol}}{dim, nrow, and ncol}}\label{dim-nrow-and-ncol}

The \texttt{dim} function will tell you how many rows (i.e., how many
observational units) and columns (i.e., how many variables) are in the
data frame (in that order). The \texttt{nrow} function will tell you how
many rows there are, and the \texttt{ncol} function will tell you how
many columns there are.

\begin{Shaded}
\begin{Highlighting}[]
\KeywordTok{dim}\NormalTok{(nhanes)}
\end{Highlighting}
\end{Shaded}

\begin{verbatim}
## [1] 5000   15
\end{verbatim}

\begin{Shaded}
\begin{Highlighting}[]
\KeywordTok{nrow}\NormalTok{(nhanes)}
\end{Highlighting}
\end{Shaded}

\begin{verbatim}
## [1] 5000
\end{verbatim}

\begin{Shaded}
\begin{Highlighting}[]
\KeywordTok{ncol}\NormalTok{(nhanes)}
\end{Highlighting}
\end{Shaded}

\begin{verbatim}
## [1] 15
\end{verbatim}

\subsubsection{c. Converting categorical variables to
factors}\label{c.-converting-categorical-variables-to-factors}

When you first read a data set in, quantitative data types will usually
be assigned the correct data type in R, but categorical variables will
typically be stored as a character data type in R. We'll need to tell R
that these are categorical variables by converting them to
\texttt{factors}. A factor is just R's name for a categorical variable.

Remember that we divide categorical variables into two sub-types:

\begin{enumerate}
\def\labelenumi{\arabic{enumi}.}
\tightlist
\item
  Nominal, where there is no specific order to the categories (for
  example think of eye color -- the categories might be blue, green,
  brown, etc., and there is no specific order to those categories)
\item
  Ordinal, where there is a specific order to the categories (for
  example think of education level -- the categories might be ``less
  than high school degree'', ``some college'', ``college degree'',
  ``graduate degree'')
\end{enumerate}

The difference in reading these into R is in whether or not we need to
specify an \texttt{ordered\ =\ TRUE} argument to the \texttt{factor}
function.

In both cases, we will use the \texttt{mutate} function to modify the
data frame. The \texttt{mutate} function will be described more later in
this document. It is in the \texttt{dplyr} package, so we need to load
that package before we can use it:

\begin{Shaded}
\begin{Highlighting}[]
\KeywordTok{library}\NormalTok{(dplyr)}
\end{Highlighting}
\end{Shaded}

\begin{verbatim}
## 
## Attaching package: 'dplyr'
\end{verbatim}

\begin{verbatim}
## The following objects are masked from 'package:stats':
## 
##     filter, lag
\end{verbatim}

\begin{verbatim}
## The following objects are masked from 'package:base':
## 
##     intersect, setdiff, setequal, union
\end{verbatim}

\paragraph{\texorpdfstring{Converting a nominal categorical variable to
a
\texttt{factor}}{Converting a nominal categorical variable to a factor}}\label{converting-a-nominal-categorical-variable-to-a-factor}

\begin{Shaded}
\begin{Highlighting}[]
\NormalTok{nhanes <-}\StringTok{ }\NormalTok{nhanes }\OperatorTok
\StringTok{    }\KeywordTok{mutate}\NormalTok{(}\DataTypeTok{Gender =} \KeywordTok{factor}\NormalTok{(Gender))}
\end{Highlighting}
\end{Shaded}

\paragraph{\texorpdfstring{Converting an ordinal categorical variable to
an \emph{ordered}
\texttt{factor}}{Converting an ordinal categorical variable to an ordered factor}}\label{converting-an-ordinal-categorical-variable-to-an-ordered-factor}

\begin{Shaded}
\begin{Highlighting}[]
\NormalTok{nhanes <-}\StringTok{ }\NormalTok{nhanes }\OperatorTok
\StringTok{    }\KeywordTok{mutate}\NormalTok{(}
        \DataTypeTok{Education =} \KeywordTok{factor}\NormalTok{(Education,}
            \DataTypeTok{levels =} \KeywordTok{c}\NormalTok{(}\StringTok{"8th Grade"}\NormalTok{, }\StringTok{"9 - 11th Grade"}\NormalTok{, }\StringTok{"High School"}\NormalTok{, }\StringTok{"Some College"}\NormalTok{, }\StringTok{"College Grad"}\NormalTok{),}
            \DataTypeTok{ordered =} \OtherTok{TRUE}\NormalTok{)}
\NormalTok{    )}
\end{Highlighting}
\end{Shaded}

For an ordinal variable, we need to add two more arguments to the call
to \texttt{factor}:

\begin{itemize}
\tightlist
\item
  specify the \texttt{levels} of the variable in order tell R what order
  they come in.
\item
  \texttt{ordered\ =\ TRUE} argument to tell R that it needs to pay
  attention to and remember the order we specified above.
\end{itemize}

\paragraph{Listing distinct values of a
variable}\label{listing-distinct-values-of-a-variable}

In order to know what to list for the possible levels of an ordinal
categorical variable, you can use the \texttt{distinct} function to list
the distinct values of the variable:

\begin{Shaded}
\begin{Highlighting}[]
\NormalTok{nhanes }\OperatorTok\StringTok{ }\KeywordTok{distinct}\NormalTok{(Education)}
\end{Highlighting}
\end{Shaded}

\begin{verbatim}
## # A tibble: 6 x 1
##   Education     
##   <ord>         
## 1 High School   
## 2 9 - 11th Grade
## 3 <NA>          
## 4 Some College  
## 5 College Grad  
## 6 8th Grade
\end{verbatim}

\subsection{3. Summarizing Categorical
Variables}\label{summarizing-categorical-variables}

It is often helpful to obtain counts of how many observational units
fall into each category of a categorical variable, or into each
combination of categories for two categorical variables. We will do this
with the \texttt{count} function:

\begin{Shaded}
\begin{Highlighting}[]
\NormalTok{nhanes }\OperatorTok\StringTok{ }\KeywordTok{count}\NormalTok{(Education)}
\end{Highlighting}
\end{Shaded}

\begin{verbatim}
## # A tibble: 6 x 2
##   Education          n
##   <ord>          <int>
## 1 8th Grade        212
## 2 9 - 11th Grade   405
## 3 High School      679
## 4 Some College    1160
## 5 College Grad    1128
## 6 <NA>            1416
\end{verbatim}

\begin{Shaded}
\begin{Highlighting}[]
\NormalTok{nhanes }\OperatorTok\StringTok{ }\KeywordTok{count}\NormalTok{(Education, Gender)}
\end{Highlighting}
\end{Shaded}

\begin{verbatim}
## # A tibble: 12 x 3
##    Education      Gender     n
##    <ord>          <fct>  <int>
##  1 8th Grade      female    91
##  2 8th Grade      male     121
##  3 9 - 11th Grade female   174
##  4 9 - 11th Grade male     231
##  5 High School    female   338
##  6 High School    male     341
##  7 Some College   female   615
##  8 Some College   male     545
##  9 College Grad   female   584
## 10 College Grad   male     544
## 11 <NA>           female   693
## 12 <NA>           male     723
\end{verbatim}

Sometimes for two variables, it's helpful to convert the summaries above
into a contingency table, with one variable in the rows and the other in
the columns. We can do this by adding on a call to the \texttt{spread}
function, which is in the \texttt{tidyr} package:

\begin{Shaded}
\begin{Highlighting}[]
\KeywordTok{library}\NormalTok{(tidyr)}
\end{Highlighting}
\end{Shaded}

\begin{verbatim}
## 
## Attaching package: 'tidyr'
\end{verbatim}

\begin{verbatim}
## The following object is masked from 'package:Matrix':
## 
##     expand
\end{verbatim}

\begin{Shaded}
\begin{Highlighting}[]
\NormalTok{nhanes }\OperatorTok
\StringTok{  }\KeywordTok{count}\NormalTok{(Education, Gender) }\OperatorTok
\StringTok{  }\KeywordTok{spread}\NormalTok{(Gender, n)}
\end{Highlighting}
\end{Shaded}

\begin{verbatim}
## # A tibble: 6 x 3
##   Education      female  male
##   <ord>           <int> <int>
## 1 8th Grade          91   121
## 2 9 - 11th Grade    174   231
## 3 High School       338   341
## 4 Some College      615   545
## 5 College Grad      584   544
## 6 <NA>              693   723
\end{verbatim}

\subsection{4. Summarizing Quantitative
Variables}\label{summarizing-quantitative-variables}

We can use the \texttt{summarize} function to calculate summaries of
quantitative variables in a data set:

\begin{Shaded}
\begin{Highlighting}[]
\NormalTok{nhanes }\OperatorTok
\StringTok{  }\KeywordTok{summarize}\NormalTok{(}
    \DataTypeTok{mean_poverty_index =} \KeywordTok{mean}\NormalTok{(Poverty, }\DataTypeTok{na.rm =} \OtherTok{TRUE}\NormalTok{),}
    \DataTypeTok{median_poverty_index =} \KeywordTok{median}\NormalTok{(Poverty, }\DataTypeTok{na.rm =} \OtherTok{TRUE}\NormalTok{),}
    \DataTypeTok{q1_poverty_index =} \KeywordTok{quantile}\NormalTok{(Poverty, }\DataTypeTok{probs =} \FloatTok{0.25}\NormalTok{, }\DataTypeTok{na.rm =} \OtherTok{TRUE}\NormalTok{),}
    \DataTypeTok{q3_poverty_index =} \KeywordTok{quantile}\NormalTok{(Poverty, }\DataTypeTok{probs =} \FloatTok{0.75}\NormalTok{, }\DataTypeTok{na.rm =} \OtherTok{TRUE}\NormalTok{),}
    \DataTypeTok{iqr_poverty_index =} \KeywordTok{IQR}\NormalTok{(Poverty, }\DataTypeTok{na.rm =} \OtherTok{TRUE}\NormalTok{),}
    \DataTypeTok{var_poverty_index =} \KeywordTok{var}\NormalTok{(Poverty, }\DataTypeTok{na.rm =} \OtherTok{TRUE}\NormalTok{),}
    \DataTypeTok{sd_poverty_index =} \KeywordTok{sd}\NormalTok{(Poverty, }\DataTypeTok{na.rm =} \OtherTok{TRUE}\NormalTok{)}
\NormalTok{  )}
\end{Highlighting}
\end{Shaded}

\begin{verbatim}
## # A tibble: 1 x 7
##   mean_poverty_index median_poverty_ind~ q1_poverty_index q3_poverty_index
##                <dbl>               <dbl>            <dbl>            <dbl>
## 1               2.76                 2.6             1.19             4.76
## # ... with 3 more variables: iqr_poverty_index <dbl>,
## #   var_poverty_index <dbl>, sd_poverty_index <dbl>
\end{verbatim}

If you don't need to worry about missing values in your data set, you
don't need the \texttt{na.rm\ =\ TRUE} part in the calls above. Note
that ordinarily, you'd probably only need to compute a couple of these
summaries.

If we want to compute these summaries separately for each level of a
categorical variable, we can \texttt{group\_by} that categorical
variable:

\begin{Shaded}
\begin{Highlighting}[]
\NormalTok{nhanes }\OperatorTok
\StringTok{  }\KeywordTok{group_by}\NormalTok{(MaritalStatus) }\OperatorTok
\StringTok{  }\KeywordTok{summarize}\NormalTok{(}
    \DataTypeTok{mean_poverty_index =} \KeywordTok{mean}\NormalTok{(Poverty, }\DataTypeTok{na.rm =} \OtherTok{TRUE}\NormalTok{),}
    \DataTypeTok{median_poverty_index =} \KeywordTok{median}\NormalTok{(Poverty, }\DataTypeTok{na.rm =} \OtherTok{TRUE}\NormalTok{),}
    \DataTypeTok{q1_poverty_index =} \KeywordTok{quantile}\NormalTok{(Poverty, }\DataTypeTok{probs =} \FloatTok{0.25}\NormalTok{, }\DataTypeTok{na.rm =} \OtherTok{TRUE}\NormalTok{),}
    \DataTypeTok{q3_poverty_index =} \KeywordTok{quantile}\NormalTok{(Poverty, }\DataTypeTok{probs =} \FloatTok{0.75}\NormalTok{, }\DataTypeTok{na.rm =} \OtherTok{TRUE}\NormalTok{),}
    \DataTypeTok{iqr_poverty_index =} \KeywordTok{IQR}\NormalTok{(Poverty, }\DataTypeTok{na.rm =} \OtherTok{TRUE}\NormalTok{),}
    \DataTypeTok{var_poverty_index =} \KeywordTok{var}\NormalTok{(Poverty, }\DataTypeTok{na.rm =} \OtherTok{TRUE}\NormalTok{),}
    \DataTypeTok{sd_poverty_index =} \KeywordTok{sd}\NormalTok{(Poverty, }\DataTypeTok{na.rm =} \OtherTok{TRUE}\NormalTok{)}
\NormalTok{  )}
\end{Highlighting}
\end{Shaded}

\begin{verbatim}
## # A tibble: 7 x 8
##   MaritalStatus mean_poverty_index median_poverty_index q1_poverty_index
##   <chr>                      <dbl>                <dbl>            <dbl>
## 1 Divorced                    2.54                 2.2              1.19
## 2 LivePartner                 2.46                 1.95             1   
## 3 Married                     3.36                 3.64             1.85
## 4 NeverMarried                2.42                 1.97             0.9 
## 5 Separated                   1.87                 1.36             0.96
## 6 Widowed                     2.23                 1.79             1.06
## 7 <NA>                        2.38                 1.88             0.93
## # ... with 4 more variables: q3_poverty_index <dbl>,
## #   iqr_poverty_index <dbl>, var_poverty_index <dbl>,
## #   sd_poverty_index <dbl>
\end{verbatim}

In this class, we will use the following summary functions:

\begin{itemize}
\tightlist
\item
  Summaries of center:

  \begin{itemize}
  \tightlist
  \item
    \texttt{mean} calculates the mean
  \item
    \texttt{median} calculates the median
  \end{itemize}
\item
  Summaries of spread:

  \begin{itemize}
  \tightlist
  \item
    \texttt{var} calculates the variance
  \item
    \texttt{sd} calculates the standard deviation
  \item
    \texttt{IQR} calculates the interquartile range
  \end{itemize}
\item
  Other:

  \begin{itemize}
  \tightlist
  \item
    \texttt{quantile(...,\ probs\ =\ 0.25)} calculates the 25th
    percentile
  \end{itemize}
\end{itemize}

\subsection{5. Data Wrangling}\label{data-wrangling}

In this class, we will learn about a few of the most common operations
you may want to perform on data sets. Here are the ones we've talked
about so far; we'll add a couple more to this list later:

\begin{itemize}
\item
  \begin{enumerate}
  \def\labelenumi{\alph{enumi}.}
  \tightlist
  \item
    Add new \textbf{variables} or modify existing \textbf{variables}
    (remember that variables correspond to columns of the data frame):
  \end{enumerate}

  \begin{itemize}
  \tightlist
  \item
    \texttt{mutate}: add a new variable or modify an existing variable
  \end{itemize}
\item
  \begin{enumerate}
  \def\labelenumi{\alph{enumi}.}
  \setcounter{enumi}{1}
  \tightlist
  \item
    Keep a subset of \textbf{observational units} (rows):
  \end{enumerate}

  \begin{itemize}
  \tightlist
  \item
    \texttt{filter}: keep only a subset of the observational units in
    the data frame that meet conditions you specify
  \end{itemize}
\item
  \begin{enumerate}
  \def\labelenumi{\alph{enumi}.}
  \setcounter{enumi}{2}
  \tightlist
  \item
    Arrange the \textbf{observational units} (rows) in order:
  \end{enumerate}

  \begin{itemize}
  \tightlist
  \item
    \texttt{arrange}: sort the observations in order according to one of
    the variables
  \end{itemize}
\end{itemize}

All of these functions are in the \texttt{dplyr} package, so we'll need
to load that package:

\begin{Shaded}
\begin{Highlighting}[]
\KeywordTok{library}\NormalTok{(dplyr)}
\end{Highlighting}
\end{Shaded}

\subsubsection{\texorpdfstring{a.
\texttt{mutate}}{a. mutate}}\label{a.-mutate}

The basic use of \texttt{mutate} looks like this:

\begin{Shaded}
\begin{Highlighting}[]
\OperatorTok{<}\NormalTok{name of modified data frame}\OperatorTok{>}\StringTok{ }\ErrorTok{<}\OperatorTok{-}\StringTok{ }\ErrorTok{<}\NormalTok{original data frame}\OperatorTok{>}\StringTok{ }\OperatorTok
\StringTok{    }\KeywordTok{mutate}\NormalTok{(}
        \OperatorTok{<}\NormalTok{new}\OperatorTok{/}\NormalTok{modified variable }\DecValTok{1}\OperatorTok{>}\StringTok{ }\ErrorTok{=}\StringTok{ }\ErrorTok{<}\NormalTok{how to calculate new}\OperatorTok{/}\NormalTok{modified variable }\DecValTok{1}\OperatorTok{>}\NormalTok{,}
        \OperatorTok{<}\NormalTok{new}\OperatorTok{/}\NormalTok{modified variable }\DecValTok{2}\OperatorTok{>}\StringTok{ }\ErrorTok{=}\StringTok{ }\ErrorTok{<}\NormalTok{how to calculate new}\OperatorTok{/}\NormalTok{modified variable }\DecValTok{2}\OperatorTok{>}
\StringTok{    }\NormalTok{)}
\end{Highlighting}
\end{Shaded}

Note that the \texttt{mutate} function does not necessarily modify the
original data frame: it creates a second copy, and leaves the original
as it was.

Suppose we want to convert the subjects' weight in kilograms to a weight
in pounds, and add the weight in pounds to the data frame as a new
variable called \texttt{Weight\_pounds}. Here's how we can do that
(there are 2.205 pounds in a kilogram):

\begin{Shaded}
\begin{Highlighting}[]
\NormalTok{nhanes_with_weight_in_pounds <-}\StringTok{ }\NormalTok{nhanes }\OperatorTok
\StringTok{    }\KeywordTok{mutate}\NormalTok{(}\DataTypeTok{Weight_pounds =}\NormalTok{ Weight }\OperatorTok{*}\StringTok{ }\FloatTok{2.205}\NormalTok{)}
\end{Highlighting}
\end{Shaded}

Here's a look at the structure of the newly created data frame,
\texttt{nhanes\_with\_weight\_in\_pounds}. Note the addition of a new
variable at the end called \texttt{Weight\_pounds}. If we were to look
at the original \texttt{nhanes} data frame, we would see that it was not
changed.

\begin{Shaded}
\begin{Highlighting}[]
\KeywordTok{str}\NormalTok{(nhanes_with_weight_in_pounds)}
\end{Highlighting}
\end{Shaded}

\begin{verbatim}
## Classes 'tbl_df', 'tbl' and 'data.frame':    5000 obs. of  16 variables:
##  $ ID           : int  3923 1548 1205 1519 4148 1681 3710 3733 2552 373 ...
##  $ Gender       : Factor w/ 2 levels "female","male": 1 2 2 2 2 1 1 2 2 1 ...
##  $ Age          : int  80 42 4 12 1 14 56 0 33 46 ...
##  $ Race         : chr  "White" "Black" "Hispanic" "Black" ...
##  $ Education    : Ord.factor w/ 5 levels "8th Grade"<"9 - 11th Grade"<..: 3 2 NA NA NA NA 4 NA 2 4 ...
##  $ MaritalStatus: chr  "Married" "LivePartner" NA NA ...
##  $ HHIncome     : chr  "55000-64999" "5000-9999" "25000-34999" "75000-99999" ...
##  $ Poverty      : num  4.27 0.3 0.78 2.96 0.67 1.52 5 2.46 NA 2.81 ...
##  $ Weight       : num  71.1 115.4 19.7 63.7 11.7 ...
##  $ Length       : num  NA NA NA NA 84.2 NA NA 61.7 NA NA ...
##  $ Height       : num  162 165 110 170 NA ...
##  $ Diabetes     : chr  "Yes" "Yes" "No" "No" ...
##  $ nPregnancies : int  5 NA NA NA NA NA 2 NA NA 3 ...
##  $ nBabies      : int  4 NA NA NA NA NA 2 NA NA 2 ...
##  $ PregnantNow  : chr  NA NA NA NA ...
##  $ Weight_pounds: num  156.8 254.5 43.4 140.5 25.8 ...
\end{verbatim}

\subsubsection{\texorpdfstring{b.
\texttt{filter}}{b. filter}}\label{b.-filter}

We often want to look at a subset of the observational units in a data
frame. The \texttt{filter} command lets us do this by specifying values
of the variables we want to keep. In this class, we will use a small
amount of the filtering capabilities that R provides. Here are a few
examples of some filters we will use. As with the \texttt{mutate}
command, \texttt{filter} does not modify the original data set.

\paragraph{Filter according to the value of a categorical
variable}\label{filter-according-to-the-value-of-a-categorical-variable}

In the command below we keep only observational units with an
\texttt{Education} level of ``High School''. Note the use of two equals
signs and quotes around the value we want to keep.

\begin{Shaded}
\begin{Highlighting}[]
\NormalTok{nhanes_fewer_obs_units <-}\StringTok{ }\NormalTok{nhanes }\OperatorTok
\StringTok{    }\KeywordTok{filter}\NormalTok{(Education }\OperatorTok{==}\StringTok{ "High School"}\NormalTok{)}
\end{Highlighting}
\end{Shaded}

\begin{Shaded}
\begin{Highlighting}[]
\KeywordTok{head}\NormalTok{(nhanes_fewer_obs_units)}
\end{Highlighting}
\end{Shaded}

\begin{verbatim}
## # A tibble: 6 x 15
##      ID Gender   Age Race  Education MaritalStatus HHIncome Poverty Weight
##   <int> <fct>  <int> <chr> <ord>     <chr>         <chr>      <dbl>  <dbl>
## 1  3923 female    80 White High Sch~ Married       55000-6~    4.27   71.1
## 2  4880 male      80 White High Sch~ Married       45000-5~    3.48   86.4
## 3  1858 female    73 White High Sch~ Married       25000-3~    1.91   91.6
## 4   181 female    80 White High Sch~ Married       35000-4~    2.64   81.6
## 5  4991 male      80 White High Sch~ Married       55000-6~    4.08   71.5
## 6  1895 male      58 Mexi~ High Sch~ Married       20000-2~    1.56   80.7
## # ... with 6 more variables: Length <dbl>, Height <dbl>, Diabetes <chr>,
## #   nPregnancies <int>, nBabies <int>, PregnantNow <chr>
\end{verbatim}

\paragraph{Filter according to the value of a categorical variable, keep
multiple
values}\label{filter-according-to-the-value-of-a-categorical-variable-keep-multiple-values}

In the command below we keep only observational units with an
\texttt{Education} level of ``High School'' or ``Some College''. Note
the use of two equals signs and quotes around the values we want to
keep. The vertical line in between the two possible values can be read
as ``or''. On my keyboard, that symbol is above the backslash, on the
right side of the keyboard.

\begin{Shaded}
\begin{Highlighting}[]
\NormalTok{nhanes_fewer_obs_units <-}\StringTok{ }\NormalTok{nhanes }\OperatorTok
\StringTok{    }\KeywordTok{filter}\NormalTok{(Education }\OperatorTok{==}\StringTok{ "High School"} \OperatorTok{|}\StringTok{ }\NormalTok{Education }\OperatorTok{==}\StringTok{ "Some College"}\NormalTok{)}
\end{Highlighting}
\end{Shaded}

\begin{Shaded}
\begin{Highlighting}[]
\KeywordTok{head}\NormalTok{(nhanes_fewer_obs_units)}
\end{Highlighting}
\end{Shaded}

\begin{verbatim}
## # A tibble: 6 x 15
##      ID Gender   Age Race  Education MaritalStatus HHIncome Poverty Weight
##   <int> <fct>  <int> <chr> <ord>     <chr>         <chr>      <dbl>  <dbl>
## 1  3923 female    80 White High Sch~ Married       55000-6~    4.27   71.1
## 2  3710 female    56 White Some Col~ Married       more 99~    5     102. 
## 3   373 female    46 White Some Col~ Divorced      45000-5~    2.81   90.9
## 4  4370 female    57 White Some Col~ Married       45000-5~    3.47   58.7
## 5  4880 male      80 White High Sch~ Married       45000-5~    3.48   86.4
## 6  1858 female    73 White High Sch~ Married       25000-3~    1.91   91.6
## # ... with 6 more variables: Length <dbl>, Height <dbl>, Diabetes <chr>,
## #   nPregnancies <int>, nBabies <int>, PregnantNow <chr>
\end{verbatim}

\paragraph{Filter according to the value of a quantitative
variable}\label{filter-according-to-the-value-of-a-quantitative-variable}

Here we keep only the observational units with an Age of at least 22:

\begin{Shaded}
\begin{Highlighting}[]
\NormalTok{nhanes_fewer_obs_units <-}\StringTok{ }\NormalTok{nhanes }\OperatorTok
\StringTok{    }\KeywordTok{filter}\NormalTok{(Age }\OperatorTok{>=}\StringTok{ }\DecValTok{22}\NormalTok{)}
\end{Highlighting}
\end{Shaded}

\begin{Shaded}
\begin{Highlighting}[]
\KeywordTok{head}\NormalTok{(nhanes_fewer_obs_units)}
\end{Highlighting}
\end{Shaded}

\begin{verbatim}
## # A tibble: 6 x 15
##      ID Gender   Age Race  Education MaritalStatus HHIncome Poverty Weight
##   <int> <fct>  <int> <chr> <ord>     <chr>         <chr>      <dbl>  <dbl>
## 1  3923 female    80 White High Sch~ Married       55000-6~    4.27   71.1
## 2  1548 male      42 Black 9 - 11th~ LivePartner   5000-99~    0.3   115. 
## 3  3710 female    56 White Some Col~ Married       more 99~    5     102. 
## 4  2552 male      33 Mexi~ 9 - 11th~ Married       <NA>       NA      90.1
## 5   373 female    46 White Some Col~ Divorced      45000-5~    2.81   90.9
## 6  4370 female    57 White Some Col~ Married       45000-5~    3.47   58.7
## # ... with 6 more variables: Length <dbl>, Height <dbl>, Diabetes <chr>,
## #   nPregnancies <int>, nBabies <int>, PregnantNow <chr>
\end{verbatim}

We could also use a variety of other conditions:

\begin{Shaded}
\begin{Highlighting}[]
\NormalTok{nhanes_fewer_obs_units <-}\StringTok{ }\NormalTok{nhanes }\OperatorTok
\StringTok{    }\KeywordTok{filter}\NormalTok{(Age }\OperatorTok{<}\StringTok{ }\DecValTok{22}\NormalTok{)}
\end{Highlighting}
\end{Shaded}

\begin{Shaded}
\begin{Highlighting}[]
\NormalTok{nhanes_fewer_obs_units <-}\StringTok{ }\NormalTok{nhanes }\OperatorTok
\StringTok{    }\KeywordTok{filter}\NormalTok{(Age }\OperatorTok{<=}\StringTok{ }\DecValTok{22}\NormalTok{)}
\end{Highlighting}
\end{Shaded}

\begin{Shaded}
\begin{Highlighting}[]
\NormalTok{nhanes_fewer_obs_units <-}\StringTok{ }\NormalTok{nhanes }\OperatorTok
\StringTok{    }\KeywordTok{filter}\NormalTok{(Age }\OperatorTok{==}\StringTok{ }\DecValTok{22}\NormalTok{)}
\end{Highlighting}
\end{Shaded}

\begin{Shaded}
\begin{Highlighting}[]
\NormalTok{nhanes_fewer_obs_units <-}\StringTok{ }\NormalTok{nhanes }\OperatorTok
\StringTok{    }\KeywordTok{filter}\NormalTok{(Age }\OperatorTok{>}\StringTok{ }\DecValTok{22}\NormalTok{)}
\end{Highlighting}
\end{Shaded}

\paragraph{Filter according to multiple
conditions}\label{filter-according-to-multiple-conditions}

If we have multiple conditions, they can be separated by commas in the
call to the filter function:

\begin{Shaded}
\begin{Highlighting}[]
\NormalTok{nhanes_fewer_obs_units <-}\StringTok{ }\NormalTok{nhanes }\OperatorTok
\StringTok{    }\KeywordTok{filter}\NormalTok{(Education }\OperatorTok{==}\StringTok{ "High School"} \OperatorTok{|}\StringTok{ }\NormalTok{Education }\OperatorTok{==}\StringTok{ "Some College"}\NormalTok{, Age }\OperatorTok{>}\StringTok{ }\DecValTok{22}\NormalTok{)}
\end{Highlighting}
\end{Shaded}

\begin{Shaded}
\begin{Highlighting}[]
\KeywordTok{head}\NormalTok{(nhanes_fewer_obs_units)}
\end{Highlighting}
\end{Shaded}

\begin{verbatim}
## # A tibble: 6 x 15
##      ID Gender   Age Race  Education MaritalStatus HHIncome Poverty Weight
##   <int> <fct>  <int> <chr> <ord>     <chr>         <chr>      <dbl>  <dbl>
## 1  3923 female    80 White High Sch~ Married       55000-6~    4.27   71.1
## 2  3710 female    56 White Some Col~ Married       more 99~    5     102. 
## 3   373 female    46 White Some Col~ Divorced      45000-5~    2.81   90.9
## 4  4370 female    57 White Some Col~ Married       45000-5~    3.47   58.7
## 5  4880 male      80 White High Sch~ Married       45000-5~    3.48   86.4
## 6  1858 female    73 White High Sch~ Married       25000-3~    1.91   91.6
## # ... with 6 more variables: Length <dbl>, Height <dbl>, Diabetes <chr>,
## #   nPregnancies <int>, nBabies <int>, PregnantNow <chr>
\end{verbatim}

\subsubsection{\texorpdfstring{c.
\texttt{arrange}}{c. arrange}}\label{c.-arrange}

The \texttt{arrange} function lets you sort the observational units in a
data frame according to the values of one of the variables.

\paragraph{Sort in ascending order (the
default)}\label{sort-in-ascending-order-the-default}

\begin{Shaded}
\begin{Highlighting}[]
\NormalTok{nhanes_sorted <-}\StringTok{ }\NormalTok{nhanes }\OperatorTok
\StringTok{    }\KeywordTok{arrange}\NormalTok{(Age)}
\end{Highlighting}
\end{Shaded}

\begin{Shaded}
\begin{Highlighting}[]
\KeywordTok{head}\NormalTok{(nhanes_sorted)}
\end{Highlighting}
\end{Shaded}

\begin{verbatim}
## # A tibble: 6 x 15
##      ID Gender   Age Race  Education MaritalStatus HHIncome Poverty Weight
##   <int> <fct>  <int> <chr> <ord>     <chr>         <chr>      <dbl>  <dbl>
## 1  3733 male       0 White <NA>      <NA>          55000-6~    2.46    6.2
## 2  2361 female     0 White <NA>      <NA>          15000-1~    0.78    5.5
## 3  1441 female     0 Hisp~ <NA>      <NA>          10000-1~    0.37    6.3
## 4  3911 female     0 White <NA>      <NA>          35000-4~    1.83    7.7
## 5  1902 male       0 White <NA>      <NA>          75000-9~    3.44    5.6
## 6  1716 female     0 White <NA>      <NA>          25000-3~    1.03    9.5
## # ... with 6 more variables: Length <dbl>, Height <dbl>, Diabetes <chr>,
## #   nPregnancies <int>, nBabies <int>, PregnantNow <chr>
\end{verbatim}

\begin{Shaded}
\begin{Highlighting}[]
\KeywordTok{head}\NormalTok{(nhanes)}
\end{Highlighting}
\end{Shaded}

\begin{verbatim}
## # A tibble: 6 x 15
##      ID Gender   Age Race  Education MaritalStatus HHIncome Poverty Weight
##   <int> <fct>  <int> <chr> <ord>     <chr>         <chr>      <dbl>  <dbl>
## 1  3923 female    80 White High Sch~ Married       55000-6~    4.27   71.1
## 2  1548 male      42 Black 9 - 11th~ LivePartner   5000-99~    0.3   115. 
## 3  1205 male       4 Hisp~ <NA>      <NA>          25000-3~    0.78   19.7
## 4  1519 male      12 Black <NA>      <NA>          75000-9~    2.96   63.7
## 5  4148 male       1 Black <NA>      <NA>          15000-1~    0.67   11.7
## 6  1681 female    14 White <NA>      <NA>          25000-3~    1.52   71.6
## # ... with 6 more variables: Length <dbl>, Height <dbl>, Diabetes <chr>,
## #   nPregnancies <int>, nBabies <int>, PregnantNow <chr>
\end{verbatim}

\paragraph{Sort in descending order}\label{sort-in-descending-order}

To sort in descending order, we wrap the variable we want to sort by in
\texttt{desc()}:

\begin{Shaded}
\begin{Highlighting}[]
\NormalTok{nhanes_sorted <-}\StringTok{ }\NormalTok{nhanes }\OperatorTok
\StringTok{    }\KeywordTok{arrange}\NormalTok{(}\KeywordTok{desc}\NormalTok{(Age))}
\end{Highlighting}
\end{Shaded}

\begin{Shaded}
\begin{Highlighting}[]
\KeywordTok{head}\NormalTok{(nhanes_sorted)}
\end{Highlighting}
\end{Shaded}

\begin{verbatim}
## # A tibble: 6 x 15
##      ID Gender   Age Race  Education MaritalStatus HHIncome Poverty Weight
##   <int> <fct>  <int> <chr> <ord>     <chr>         <chr>      <dbl>  <dbl>
## 1  3923 female    80 White High Sch~ Married       55000-6~    4.27   71.1
## 2  4880 male      80 White High Sch~ Married       45000-5~    3.48   86.4
## 3   181 female    80 White High Sch~ Married       35000-4~    2.64   81.6
## 4  4991 male      80 White High Sch~ Married       55000-6~    4.08   71.5
## 5   244 male      80 White College ~ NeverMarried  35000-4~    3.21   72.1
## 6  3617 male      80 White College ~ Married       25000-3~    1.98   72.6
## # ... with 6 more variables: Length <dbl>, Height <dbl>, Diabetes <chr>,
## #   nPregnancies <int>, nBabies <int>, PregnantNow <chr>
\end{verbatim}

\begin{Shaded}
\begin{Highlighting}[]
\KeywordTok{head}\NormalTok{(nhanes)}
\end{Highlighting}
\end{Shaded}

\begin{verbatim}
## # A tibble: 6 x 15
##      ID Gender   Age Race  Education MaritalStatus HHIncome Poverty Weight
##   <int> <fct>  <int> <chr> <ord>     <chr>         <chr>      <dbl>  <dbl>
## 1  3923 female    80 White High Sch~ Married       55000-6~    4.27   71.1
## 2  1548 male      42 Black 9 - 11th~ LivePartner   5000-99~    0.3   115. 
## 3  1205 male       4 Hisp~ <NA>      <NA>          25000-3~    0.78   19.7
## 4  1519 male      12 Black <NA>      <NA>          75000-9~    2.96   63.7
## 5  4148 male       1 Black <NA>      <NA>          15000-1~    0.67   11.7
## 6  1681 female    14 White <NA>      <NA>          25000-3~    1.52   71.6
## # ... with 6 more variables: Length <dbl>, Height <dbl>, Diabetes <chr>,
## #   nPregnancies <int>, nBabies <int>, PregnantNow <chr>
\end{verbatim}


\end{document}
