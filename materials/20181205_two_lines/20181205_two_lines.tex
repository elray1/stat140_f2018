\documentclass[14pt]{extarticle}
\usepackage{lmodern}
\usepackage{amssymb,amsmath}
\usepackage{ifxetex,ifluatex}
\usepackage{fixltx2e} % provides \textsubscript
\ifnum 0\ifxetex 1\fi\ifluatex 1\fi=0 % if pdftex
  \usepackage[T1]{fontenc}
  \usepackage[utf8]{inputenc}
\else % if luatex or xelatex
  \ifxetex
    \usepackage{mathspec}
  \else
    \usepackage{fontspec}
  \fi
  \defaultfontfeatures{Ligatures=TeX,Scale=MatchLowercase}
\fi
% use upquote if available, for straight quotes in verbatim environments
\IfFileExists{upquote.sty}{\usepackage{upquote}}{}
% use microtype if available
\IfFileExists{microtype.sty}{%
\usepackage{microtype}
\UseMicrotypeSet[protrusion]{basicmath} % disable protrusion for tt fonts
}{}
\usepackage[margin=0.6in]{geometry}
\usepackage{hyperref}
\hypersetup{unicode=true,
            pdftitle={Two Lines: Crocodiles!!},
            pdfborder={0 0 0},
            breaklinks=true}
\urlstyle{same}  % don't use monospace font for urls
\usepackage{color}
\usepackage{fancyvrb}
\newcommand{\VerbBar}{|}
\newcommand{\VERB}{\Verb[commandchars=\\\{\}]}
\DefineVerbatimEnvironment{Highlighting}{Verbatim}{commandchars=\\\{\}}
% Add ',fontsize=\small' for more characters per line
\usepackage{framed}
\definecolor{shadecolor}{RGB}{248,248,248}
\newenvironment{Shaded}{\begin{snugshade}}{\end{snugshade}}
\newcommand{\KeywordTok}[1]{\textcolor[rgb]{0.13,0.29,0.53}{\textbf{#1}}}
\newcommand{\DataTypeTok}[1]{\textcolor[rgb]{0.13,0.29,0.53}{#1}}
\newcommand{\DecValTok}[1]{\textcolor[rgb]{0.00,0.00,0.81}{#1}}
\newcommand{\BaseNTok}[1]{\textcolor[rgb]{0.00,0.00,0.81}{#1}}
\newcommand{\FloatTok}[1]{\textcolor[rgb]{0.00,0.00,0.81}{#1}}
\newcommand{\ConstantTok}[1]{\textcolor[rgb]{0.00,0.00,0.00}{#1}}
\newcommand{\CharTok}[1]{\textcolor[rgb]{0.31,0.60,0.02}{#1}}
\newcommand{\SpecialCharTok}[1]{\textcolor[rgb]{0.00,0.00,0.00}{#1}}
\newcommand{\StringTok}[1]{\textcolor[rgb]{0.31,0.60,0.02}{#1}}
\newcommand{\VerbatimStringTok}[1]{\textcolor[rgb]{0.31,0.60,0.02}{#1}}
\newcommand{\SpecialStringTok}[1]{\textcolor[rgb]{0.31,0.60,0.02}{#1}}
\newcommand{\ImportTok}[1]{#1}
\newcommand{\CommentTok}[1]{\textcolor[rgb]{0.56,0.35,0.01}{\textit{#1}}}
\newcommand{\DocumentationTok}[1]{\textcolor[rgb]{0.56,0.35,0.01}{\textbf{\textit{#1}}}}
\newcommand{\AnnotationTok}[1]{\textcolor[rgb]{0.56,0.35,0.01}{\textbf{\textit{#1}}}}
\newcommand{\CommentVarTok}[1]{\textcolor[rgb]{0.56,0.35,0.01}{\textbf{\textit{#1}}}}
\newcommand{\OtherTok}[1]{\textcolor[rgb]{0.56,0.35,0.01}{#1}}
\newcommand{\FunctionTok}[1]{\textcolor[rgb]{0.00,0.00,0.00}{#1}}
\newcommand{\VariableTok}[1]{\textcolor[rgb]{0.00,0.00,0.00}{#1}}
\newcommand{\ControlFlowTok}[1]{\textcolor[rgb]{0.13,0.29,0.53}{\textbf{#1}}}
\newcommand{\OperatorTok}[1]{\textcolor[rgb]{0.81,0.36,0.00}{\textbf{#1}}}
\newcommand{\BuiltInTok}[1]{#1}
\newcommand{\ExtensionTok}[1]{#1}
\newcommand{\PreprocessorTok}[1]{\textcolor[rgb]{0.56,0.35,0.01}{\textit{#1}}}
\newcommand{\AttributeTok}[1]{\textcolor[rgb]{0.77,0.63,0.00}{#1}}
\newcommand{\RegionMarkerTok}[1]{#1}
\newcommand{\InformationTok}[1]{\textcolor[rgb]{0.56,0.35,0.01}{\textbf{\textit{#1}}}}
\newcommand{\WarningTok}[1]{\textcolor[rgb]{0.56,0.35,0.01}{\textbf{\textit{#1}}}}
\newcommand{\AlertTok}[1]{\textcolor[rgb]{0.94,0.16,0.16}{#1}}
\newcommand{\ErrorTok}[1]{\textcolor[rgb]{0.64,0.00,0.00}{\textbf{#1}}}
\newcommand{\NormalTok}[1]{#1}
\usepackage{graphicx,grffile}
\makeatletter
\def\maxwidth{\ifdim\Gin@nat@width>\linewidth\linewidth\else\Gin@nat@width\fi}
\def\maxheight{\ifdim\Gin@nat@height>\textheight\textheight\else\Gin@nat@height\fi}
\makeatother
% Scale images if necessary, so that they will not overflow the page
% margins by default, and it is still possible to overwrite the defaults
% using explicit options in \includegraphics[width, height, ...]{}
\setkeys{Gin}{width=\maxwidth,height=\maxheight,keepaspectratio}
\IfFileExists{parskip.sty}{%
\usepackage{parskip}
}{% else
\setlength{\parindent}{0pt}
\setlength{\parskip}{6pt plus 2pt minus 1pt}
}
\setlength{\emergencystretch}{3em}  % prevent overfull lines
\providecommand{\tightlist}{%
  \setlength{\itemsep}{0pt}\setlength{\parskip}{0pt}}
\setcounter{secnumdepth}{0}
% Redefines (sub)paragraphs to behave more like sections
\ifx\paragraph\undefined\else
\let\oldparagraph\paragraph
\renewcommand{\paragraph}[1]{\oldparagraph{#1}\mbox{}}
\fi
\ifx\subparagraph\undefined\else
\let\oldsubparagraph\subparagraph
\renewcommand{\subparagraph}[1]{\oldsubparagraph{#1}\mbox{}}
\fi

%%% Use protect on footnotes to avoid problems with footnotes in titles
\let\rmarkdownfootnote\footnote%
\def\footnote{\protect\rmarkdownfootnote}

%%% Change title format to be more compact
\usepackage{titling}

% Create subtitle command for use in maketitle
\newcommand{\subtitle}[1]{
  \posttitle{
    \begin{center}\large#1\end{center}
    }
}

\setlength{\droptitle}{-2em}

  \title{Two Lines: Crocodiles!!}
    \pretitle{\vspace{\droptitle}\centering\huge}
  \posttitle{\par}
    \author{}
    \preauthor{}\postauthor{}
    \date{}
    \predate{}\postdate{}
  
\usepackage{soul}
\usepackage{booktabs}

\begin{document}
\maketitle

\begin{Shaded}
\begin{Highlighting}[]
\KeywordTok{head}\NormalTok{(crocs)}
\end{Highlighting}
\end{Shaded}

\begin{verbatim}
##    Body Head    Species
## 1 338.0 52.0     Indian
## 2 333.0 48.0 Australian
## 3 202.0 38.3     Indian
## 4 406.0 52.0 Australian
## 5 459.4 60.5 Australian
## 6 264.0 49.0     Indian
\end{verbatim}

\begin{Shaded}
\begin{Highlighting}[]
\KeywordTok{ggplot}\NormalTok{(}\DataTypeTok{data =}\NormalTok{ crocs, }\DataTypeTok{mapping =} \KeywordTok{aes}\NormalTok{(}\DataTypeTok{x =}\NormalTok{ Body, }\DataTypeTok{y =}\NormalTok{ Head, }\DataTypeTok{color =}\NormalTok{ Species)) }\OperatorTok{+}
\StringTok{  }\KeywordTok{geom_point}\NormalTok{() }\OperatorTok{+}
\StringTok{  }\KeywordTok{geom_smooth}\NormalTok{(}\DataTypeTok{method =} \StringTok{"lm"}\NormalTok{, }\DataTypeTok{se =} \OtherTok{FALSE}\NormalTok{)}
\end{Highlighting}
\end{Shaded}

\includegraphics{20181205_two_lines_files/figure-latex/unnamed-chunk-3-1.pdf}

\subsubsection{2 lines by filtering to create separate data
sets}\label{lines-by-filtering-to-create-separate-data-sets}

\begin{Shaded}
\begin{Highlighting}[]
\NormalTok{aus_crocs <-}\StringTok{ }\NormalTok{crocs }\OperatorTok\StringTok{ }\KeywordTok{filter}\NormalTok{(Species }\OperatorTok{==}\StringTok{ "Australian"}\NormalTok{)}
\NormalTok{aus_fit <-}\StringTok{ }\KeywordTok{lm}\NormalTok{(Head }\OperatorTok{~}\StringTok{ }\NormalTok{Body, }\DataTypeTok{data =}\NormalTok{ aus_crocs)}
\KeywordTok{summary}\NormalTok{(aus_fit)}
\end{Highlighting}
\end{Shaded}

\begin{verbatim}
## 
## Call:
## lm(formula = Head ~ Body, data = aus_crocs)
## 
## Residuals:
##     Min      1Q  Median      3Q     Max 
## -2.3529 -0.9968  0.0824  0.7419  2.7973 
## 
## Coefficients:
##             Estimate Std. Error t value Pr(>|t|)    
## (Intercept) 3.463022   1.523732   2.273   0.0407 *  
## Body        0.125344   0.004819  26.010 1.35e-12 ***
## ---
## Signif. codes:  0 '***' 0.001 '**' 0.01 '*' 0.05 '.' 0.1 ' ' 1
## 
## Residual standard error: 1.504 on 13 degrees of freedom
## Multiple R-squared:  0.9811, Adjusted R-squared:  0.9797 
## F-statistic: 676.5 on 1 and 13 DF,  p-value: 1.35e-12
\end{verbatim}

\begin{Shaded}
\begin{Highlighting}[]
\NormalTok{ind_crocs <-}\StringTok{ }\NormalTok{crocs }\OperatorTok\StringTok{ }\KeywordTok{filter}\NormalTok{(Species }\OperatorTok{==}\StringTok{ "Indian"}\NormalTok{)}
\NormalTok{ind_fit <-}\StringTok{ }\KeywordTok{lm}\NormalTok{(Head }\OperatorTok{~}\StringTok{ }\NormalTok{Body, }\DataTypeTok{data =}\NormalTok{ ind_crocs)}
\KeywordTok{summary}\NormalTok{(ind_fit)}
\end{Highlighting}
\end{Shaded}

\begin{verbatim}
## 
## Call:
## lm(formula = Head ~ Body, data = ind_crocs)
## 
## Residuals:
##     Min      1Q  Median      3Q     Max 
## -4.5756 -1.6627 -0.0904  1.2208  4.6261 
## 
## Coefficients:
##              Estimate Std. Error t value Pr(>|t|)    
## (Intercept) 10.538438   1.861787    5.66 4.53e-05 ***
## Body         0.131304   0.005791   22.68 5.08e-13 ***
## ---
## Signif. codes:  0 '***' 0.001 '**' 0.01 '*' 0.05 '.' 0.1 ' ' 1
## 
## Residual standard error: 2.503 on 15 degrees of freedom
## Multiple R-squared:  0.9717, Adjusted R-squared:  0.9698 
## F-statistic: 514.2 on 1 and 15 DF,  p-value: 5.08e-13
\end{verbatim}

\paragraph{Questions we'd like to be able to answer (but can't with this
output):}\label{questions-wed-like-to-be-able-to-answer-but-cant-with-this-output}

\begin{enumerate}
\def\labelenumi{\arabic{enumi}.}
\tightlist
\item
  Is there statistically significant evidence that the intercepts for
  these lines are different?
\item
  Is there statistically significant evidence that the slopes for these
  lines are different?
\end{enumerate}

\newpage

\subsubsection{2 parallel lines (same
slope)}\label{parallel-lines-same-slope}

\begin{itemize}
\tightlist
\item
  Our Goal: Equations for two lines

  \begin{align*}
  \text{Predicted Head Length for Australian Crocs } &= \hat{\beta}_0^{Australian} + \hat{\beta}_{1} \times (\text{Body Length}) \\
  \text{Predicted Head Length for Indian Crocs } &= \hat{\beta}_0^{Indian} + \hat{\beta}_{1} \times (\text{Body Length})
  \end{align*}
\item
  Note: Different intercepts, same slope.
\end{itemize}

\begin{Shaded}
\begin{Highlighting}[]
\NormalTok{parallel_lines_fit <-}\StringTok{ }\KeywordTok{lm}\NormalTok{(Head }\OperatorTok{~}\StringTok{ }\NormalTok{Body }\OperatorTok{+}\StringTok{ }\NormalTok{Species, }\DataTypeTok{data =}\NormalTok{ crocs)}
\KeywordTok{summary}\NormalTok{(parallel_lines_fit)}
\end{Highlighting}
\end{Shaded}

\begin{verbatim}
## 
## Call:
## lm(formula = Head ~ Body + Species, data = crocs)
## 
## Residuals:
##     Min      1Q  Median      3Q     Max 
## -4.4959 -1.4218 -0.0842  1.0117  4.6405 
## 
## Coefficients:
##               Estimate Std. Error t value Pr(>|t|)    
## (Intercept)   2.265418   1.309167    1.73   0.0942 .  
## Body          0.129261   0.003904   33.11  < 2e-16 ***
## SpeciesIndian 8.893772   0.737538   12.06 8.05e-13 ***
## ---
## Signif. codes:  0 '***' 0.001 '**' 0.01 '*' 0.05 '.' 0.1 ' ' 1
## 
## Residual standard error: 2.082 on 29 degrees of freedom
## Multiple R-squared:  0.977,  Adjusted R-squared:  0.9755 
## F-statistic:   617 on 2 and 29 DF,  p-value: < 2.2e-16
\end{verbatim}

\begin{itemize}
\tightlist
\item
  R gives us a single combined equation:
  \[\text{Predicted Head Length} = \hat{\beta}_0 + \hat{\beta}_{1} (\text{Body Length}) + \hat{\beta}_2 \text{SpeciesIndian}\]
\end{itemize}

\[\text{Predicted Head Length} = 2.27 + 0.13 (\text{Body Length}) + 8.89 \text{SpeciesIndian}\]

\newpage

\paragraph{\texorpdfstring{What is the \texttt{SpeciesIndian}
variable?}{What is the SpeciesIndian variable?}}\label{what-is-the-speciesindian-variable}

\begin{itemize}
\item
  Behind the scenes, R creates a new \textbf{indicator variable} called
  \texttt{SpeciesIndian}:
  \[\text{{\tt SpeciesIndian}} = \begin{cases} 1 & \text{ if the species for crocodile $i$ is Indian.} \\
  0 & \text{ otherwise (in this case, the species is Australian)} \end{cases}
  \]
\item
  R doesn't modify the data frame (it creates a secret copy in the
  background), but it would look like this:
\end{itemize}

\begin{Shaded}
\begin{Highlighting}[]
\KeywordTok{head}\NormalTok{(crocs)}
\end{Highlighting}
\end{Shaded}

\begin{verbatim}
##    Body Head    Species SpeciesIndian
## 1 338.0 52.0     Indian             1
## 2 333.0 48.0 Australian             0
## 3 202.0 38.3     Indian             1
## 4 406.0 52.0 Australian             0
## 5 459.4 60.5 Australian             0
## 6 264.0 49.0     Indian             1
\end{verbatim}

Above, we obtained this estimated equation:

\[\text{Predicted Head Length} = 2.27 + 0.13 (\text{Body Length}) + 8.89 \text{SpeciesIndian}\]

\paragraph{What is the estimated equation describing the relationship
between body length and head length, for Australian
crocodiles?}\label{what-is-the-estimated-equation-describing-the-relationship-between-body-length-and-head-length-for-australian-crocodiles}

\vspace{3cm}

\paragraph{What is the estimated equation describing the relationship
between body length and head length, for Indian
crocodiles?}\label{what-is-the-estimated-equation-describing-the-relationship-between-body-length-and-head-length-for-indian-crocodiles}

\vspace{3cm}

\paragraph{\texorpdfstring{What is the interpretation of
\(\widehat{\beta}_0 = 2.27\)?}{What is the interpretation of \textbackslash{}widehat\{\textbackslash{}beta\}\_0 = 2.27?}}\label{what-is-the-interpretation-of-widehatbeta_0-2.27}

\vspace{3cm}

\paragraph{\texorpdfstring{What is the interpretation of
\(\widehat{\beta}_1 = 0.13\)?}{What is the interpretation of \textbackslash{}widehat\{\textbackslash{}beta\}\_1 = 0.13?}}\label{what-is-the-interpretation-of-widehatbeta_1-0.13}

\vspace{3cm}

\paragraph{\texorpdfstring{What is the interpretation of
\(\widehat{\beta}_2 = 8.89\)?}{What is the interpretation of \textbackslash{}widehat\{\textbackslash{}beta\}\_2 = 8.89?}}\label{what-is-the-interpretation-of-widehatbeta_2-8.89}

\vspace{3cm}

\paragraph{Using the output from the summary function, conduct a test of
the claim that the intercept of the line describing the relationship
between body length and head length in the population of all Australian
crocodiles is the same as the intercept of the line describing the
relationship between body length and head length in the population of
all Indian
crocodiles.}\label{using-the-output-from-the-summary-function-conduct-a-test-of-the-claim-that-the-intercept-of-the-line-describing-the-relationship-between-body-length-and-head-length-in-the-population-of-all-australian-crocodiles-is-the-same-as-the-intercept-of-the-line-describing-the-relationship-between-body-length-and-head-length-in-the-population-of-all-indian-crocodiles.}

\newpage

\subsubsection{Lines with different slopes
(interactions)}\label{lines-with-different-slopes-interactions}

To get different slopes, use \texttt{Body\ *\ Species} instead of
\texttt{Body\ +\ Species}:

\begin{Shaded}
\begin{Highlighting}[]
\NormalTok{two_lines_fit <-}\StringTok{ }\KeywordTok{lm}\NormalTok{(Head }\OperatorTok{~}\StringTok{ }\NormalTok{Species }\OperatorTok{*}\StringTok{ }\NormalTok{Body, }\DataTypeTok{data =}\NormalTok{ crocs)}
\KeywordTok{summary}\NormalTok{(two_lines_fit)}
\end{Highlighting}
\end{Shaded}

\begin{verbatim}
## 
## Call:
## lm(formula = Head ~ Species * Body, data = crocs)
## 
## Residuals:
##     Min      1Q  Median      3Q     Max 
## -4.5756 -1.3294 -0.0040  0.9646  4.6261 
## 
## Coefficients:
##                    Estimate Std. Error t value Pr(>|t|)    
## (Intercept)        3.463022   2.126572   1.628   0.1146    
## SpeciesIndian      7.075415   2.638253   2.682   0.0121 *  
## Body               0.125344   0.006726  18.637   <2e-16 ***
## SpeciesIndian:Body 0.005959   0.008296   0.718   0.4785    
## ---
## Signif. codes:  0 '***' 0.001 '**' 0.01 '*' 0.05 '.' 0.1 ' ' 1
## 
## Residual standard error: 2.099 on 28 degrees of freedom
## Multiple R-squared:  0.9775, Adjusted R-squared:  0.975 
## F-statistic: 404.6 on 3 and 28 DF,  p-value: < 2.2e-16
\end{verbatim}

\paragraph{What is the estimated equation from this
model?}\label{what-is-the-estimated-equation-from-this-model}

\vspace{3cm}

\paragraph{What is the estimated equation describing the relationship
between body length and head length, for Australian
crocodiles?}\label{what-is-the-estimated-equation-describing-the-relationship-between-body-length-and-head-length-for-australian-crocodiles-1}

\vspace{3cm}

\paragraph{What is the estimated equation describing the relationship
between body length and head length, for Indian
crocodiles?}\label{what-is-the-estimated-equation-describing-the-relationship-between-body-length-and-head-length-for-indian-crocodiles-1}

\vspace{3cm}

\paragraph{\texorpdfstring{What is the interpretation of
\(\widehat{\beta}_0 = 3.463\)?}{What is the interpretation of \textbackslash{}widehat\{\textbackslash{}beta\}\_0 = 3.463?}}\label{what-is-the-interpretation-of-widehatbeta_0-3.463}

\vspace{3cm}

\paragraph{\texorpdfstring{What is the interpretation of
\(\widehat{\beta}_1 = 7.075\)?}{What is the interpretation of \textbackslash{}widehat\{\textbackslash{}beta\}\_1 = 7.075?}}\label{what-is-the-interpretation-of-widehatbeta_1-7.075}

\vspace{3cm}

\paragraph{\texorpdfstring{What is the interpretation of
\(\widehat{\beta}_2 = 0.125\)?}{What is the interpretation of \textbackslash{}widehat\{\textbackslash{}beta\}\_2 = 0.125?}}\label{what-is-the-interpretation-of-widehatbeta_2-0.125}

\vspace{3cm}

\paragraph{\texorpdfstring{What is the interpretation of
\(\widehat{\beta}_3 = 0.006\)?}{What is the interpretation of \textbackslash{}widehat\{\textbackslash{}beta\}\_3 = 0.006?}}\label{what-is-the-interpretation-of-widehatbeta_3-0.006}

\newpage

\paragraph{\texorpdfstring{Using the output from the summary function,
conduct a test of the claim that the \emph{slope} of the line describing
the relationship between body length and head length in the population
of all Australian crocodiles is the same as the \emph{slope} of the line
describing the relationship between body length and head length in the
population of all Indian
crocodiles.}{Using the output from the summary function, conduct a test of the claim that the slope of the line describing the relationship between body length and head length in the population of all Australian crocodiles is the same as the slope of the line describing the relationship between body length and head length in the population of all Indian crocodiles.}}\label{using-the-output-from-the-summary-function-conduct-a-test-of-the-claim-that-the-slope-of-the-line-describing-the-relationship-between-body-length-and-head-length-in-the-population-of-all-australian-crocodiles-is-the-same-as-the-slope-of-the-line-describing-the-relationship-between-body-length-and-head-length-in-the-population-of-all-indian-crocodiles.}


\end{document}
